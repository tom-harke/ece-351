\documentclass{article}
\usepackage{circuitikz}
\begin{document}

\section{Problem 1 A}
operationally a tri signal is identical to a wire signal

the use of keyword 'tri' serves as documentation of the intent of the author to have z as a valid level.

\begin{verbatim}
wire in, c;
tri out;

assign out = c ? in : z;
\end{verbatim}

\section{Problem 1 B}


WARNING -- this is not the answer, it's a demo of using a library
\begin{circuitikz} \draw
(0,2) node[and port] (myand1) {}
(0,0) node[and port] (myand2) {}
(2,1) node[xnor port] (myxnor) {}
(myand1.out) -- (myxnor.in 1)
(myand2.out) -- (myxnor.in 2);
\end{circuitikz}


\section{Problem 1 C}

\begin{verbatim}
module hw1_prob1c (
  input logic A, B, C, D, E, F, G,
  output logic Y
);

  logic X;

  assign X = not( not( not(A & B & C) & D ) | not( E | F & G ) );
  assign Y = not(x and x);
endmodule: hw1_prob1c
\end{verbatim}
\section{Etc}
Thanks to  \verb'circuitikz'!
\end{document}
