\documentclass{article}
\usepackage{circuitikz}
\usepackage{fancyvrb}
\begin{document}

\section{Problem 1}
\subsection{Problem 1 A}
Operationally a tri signal is identical to a wire signal.
The use of keyword 'tri' serves as documentation of the intent of the author to have z as a valid level.

\begin{verbatim}
wire in, c;
tri out;

assign out = c ? in : 1'bz;
\end{verbatim}

\subsection{Problem 1 B}

\begin{verbatim}
module hw1_prob1b (
  input logic a, b, c,
  output logic y, z
);
  assign y = a & b & c | a & b & ~c | a & ~b & c;
  assign z = a & b | ~a & ~b;
endmodule: hw1_prob1b
\end{verbatim}

Warning! The inverted signals are not drawn correctly (this is my first
time using \verb'circuitikz')

\begin{center}
\begin{circuitikz} \draw
(0,7) node[above](a){$a$} 
(2,7) node[above](b){$b$} 
(4,7) node[above](c){$c$} 
(1,6) node[not port](anot){}
      (a) |- (anot.in 1)
(3,6) node[not port](bnot){}
      (b) |- (bnot.in 1)
(5,6) node[not port](cnot){}
      (c) |- (cnot.in 1)
(8,4) node[and port, number inputs=3] (yand1) {}
      (a) |- (yand1.in 3)
      (b) |- (yand1.in 2)
      (c) |- (yand1.in 1)
(8,2) node[and port, number inputs=3] (yand2) {}
      (a)    |- (yand2.in 3)
      (b)    |- (yand2.in 2)
      (cnot) |- (yand2.in 1)
(8,0) node[and port, number inputs=3] (yand3) {}
      (a)    |- (yand3.in 3)
      (bnot) |- (yand3.in 2)
      (c)    |- (yand3.in 1)
(10,2) node[or port, number inputs=3] (yor) {}
(11,2) node[above](y){$y$} 
(yand1.out) -| (yor.in 1)
(yand2.out) -| (yor.in 2)
(yand3.out) -| (yor.in 3)
(yor.out) -- (y)
(8,-2) node[and port] (zand1) {}
       (a) |- (zand1.in 1)
       (b) |- (zand1.in 2)
(8,-4) node[and port] (zand2) {}
       (anot) |- (zand2.in 1)
       (bnot) |- (zand2.in 2)
(10,-3) node[or port] (zor) {}
(11,-3) node[above](z){$z$} 
(zand1.out) -| (zor.in 1)
(zand2.out) -| (zor.in 2)
(zor.out) -- (z)
;
\end{circuitikz}
\end{center}


\subsection{C}

\begin{Verbatim}[frame=single]
module hw1_prob1c (
  input logic A, B, C, D, E, F, G,
  output logic Y
);

  logic X;

  assign X = ~( ~( ~(A & B & C) & D ) | ~( E | F & G ) );
  assign Y = ~(x & x);
endmodule: hw1_prob1c
\end{Verbatim}
\section{Problem 2: 32-bit ALU}

I did not get this done completely.  Read on for details.

I took the strategy of parameterizing the design by the width of the word.
Instantiating at 32 gives the desired ALU, but for part of the testing
we can instantiate to a much smaller width and quickly do exhaustive testing
of that smaller design.

The reasoning behind this is that, of the 32 bits, the lower order
bits are completely predictable/uninteresting.  The places where the
design is most likely to have obscure bugs is the treatment of the
highest-order bit with the flags.

I could have made a mistake implementing the parametrization, so there's still
need to do some selected tests on the full 32-bit design -- this is where my time/knowledge ran out --
I was unable to figure out how to get 2 instances of the ALU living in the testbench.

The testbench is, I believe, self-checking -- other than the start and end messages it only prints when an error is found:
\begin{verbatim}
# vector         126 is 011111100010000
# op is 01, a is 111(7), b is 110(6)
# expect: 001(1) with flags 0000
# got   : 001(1) with flags 0010
# so far:          91 passed,          36 failed
\end{verbatim}
{document}

Also, when the testbench check AND/OR it ignores the arithmetic flags.

%My testbench has $2^3 \time 2^3$ vectors, and initially several of my vectors were wrong.
%I found that I didn't understand the design as well as I originally thought.
%I would have liked to have spent more time
% arizonans = Aseurasians
