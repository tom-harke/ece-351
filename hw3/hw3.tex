\documentclass{article}
\usepackage{circuitikz}
\usepackage{fancyvrb}
\usepackage{enumitem}
\usepackage{pdfpages}
\usepackage{multicol}
%\usepackage[hscale=0.7,vscale=0.8]{geometry}
\usepackage[margin=1.5cm]{geometry}


\setlength{\parindent}{0pt}

\title{ECE 351 HW 3}
\date{2024 Nov 27}
\author{Tom Harke}
%\setlength{\columnseprule}{.4pt}


\begin{document}
\maketitle

\begin{multicols}{2}

\section{Design}
I'm not using the suggested FSM for two reasons
\begin{enumerate}[noitemsep]
\item it has far more states that seem necessary
\item the descriptions don't match how I view the machine
\end{enumerate}
Instead I'll try to motivate a simpler 3-state FSM by
examining the constraints on how the CPU moves data
around -- specifically, focussing on what needs to be
registered, and leaving the rest to combinational logic.

The longest chain of register updates is for processsing
a binary operation.
For concreteness we'll focus on 4+3, as all other instances
are isomorphic (modulo error detection due
to divide-by-zero or overflow).

The operation 4+3 arises in the context of the next op being
\verb'ADD' and with a stack of the form:
\begin{center}
	$4:3:\sigma$
\end{center}
where $4$ is at the top of the stack, $3$ is next-to-top,
and $\sigma$ represents the rest of the stack

In this context the stack changes state
\begin{center} $4:3:\sigma$ \end{center}
\begin{center}   $3:\sigma$ \end{center}
\begin{center}     $\sigma$ \end{center}
\begin{center}   $7:\sigma$ \end{center}

We need to have each change on a clock edge so we can pop or push.
\section{Finite State Machine}\
TODO
\section{Timing}

All binary ops ($+$,$-$,$*$,$/$,$\%$,$\&$,$|$)
follow this timing example.

The stack is included, even though it's more than a register,
using the notation from above (eg $3:4:\sigma$).
\includegraphics[scale=1.0]{add.timing.pdf}
The \verb'invert' op follows this timing example
\includegraphics[scale=1.0]{inv.timing.pdf}
The \verb'push_immediate' op follows this timing example
\includegraphics[scale=1.0]{imm.timing.pdf}
\end{multicols}

\end{document}
