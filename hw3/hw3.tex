\documentclass{article}
\usepackage{circuitikz}
\usepackage{fancyvrb}
\usepackage{enumitem}
\usepackage{pdfpages}
\usepackage{multicol}
%\usepackage[hscale=0.7,vscale=0.8]{geometry}
\usepackage[margin=1.5cm]{geometry}


\setlength{\parindent}{0pt}

\title{ECE 351 HW 3}
\date{2024 Nov 27}
\author{Tom Harke}
%\setlength{\columnseprule}{.4pt}


\begin{document}
\maketitle

%\begin{multicols}{2}

\section{Design}
\subsection{Finite State Machine}
\includegraphics[scale=0.4]{fsm.pdf}
\subsection{Fewer States}
I tried to merge a number of states from the suggested FSM,
and slightly changed the meaning of the POP/PUSH states.

I originally thought I could get away with 3 or 4 states, but kept encountering problems.
When I expanded to 5,
I finally got a machine that worked correctly, but the design may still show the scars of trying to minimize the states.


%The reason for merging is that I suspect we don't need that many
%	-- we only need enough so that we can sequence the flow of separate pieces of stack data on separate clock cycle.
With my FSM the states are:
\begin{description}[noitemsep]
\item[FETCH:]  get the next instruction
\item[DECODE:] start taking action (choose next state, signal a pop if needed)
\item[POP2:]   means the 2nd argument (if any) has already been popped off the stack
\item[POP1:]   means the 1st argument (if any) has already been popped off the stack
\item[PUSH:]   the result is computed and ready to push onto the stack
\item[ERROR:]  something has gone wrong. (This is not in the diagram)
\end{description}


%Note that each of these is in the past tense.
%
%I'll try to motivate a simpler 3-state FSM by
%examining the constraints on how the CPU moves data
%around -- specifically, focussing on what needs to be
%registered, and leaving the rest to combinational logic.
%
%The longest chain of register updates is for processsing
%a binary operation.

\subsection{Removal of HALT State}

I'm not sure how I did it \verb':-)',
but my design doesn't need \verb'HALT' to terminate.
So I've removed it from all .mem files and from the
verilog.

\subsection{Error Checking}
I've implemented checks for:
\begin{itemize}[noitemsep]
\item divide-by-zero
\item push to full stack
\item pop from empty stack
\end{itemize}

I did not have time to add error handling for the following problems
\begin{itemize}[noitemsep]
\item overflow
\item a final stack having more than 1 entry.
\end{itemize}

To minimize clutter the ERROR state is not depicted in the FSM diagram as there are several transitions to it,
and all are terminal.
\section{Timing}
My design was driven by my expected timing diagrams.
My reasoning was that the main constraint in the
design was that we needed the clock signal to sequence
the pushes and pops, and to stash the instruction
and the popped values into registers.
All other computation could be done via combinational
logic.

These are not the original designs that I made, but
they've been tweaked as I've debugged my design.

\subsection{Binary Operations}

All binary ops (\verb'+',\verb'-',\verb'*',\verb'/',\verb'\%',\verb'\&',\verb'|') follow the timing diagram below.
For concreteness we'll focus on 4+3, as all other instances
are isomorphic (modulo error detection due to divide-by-zero or overflow).

\begin{center}
\includegraphics[scale=1.0]{add.timing.pdf}
\end{center}

The state of the stack is included in the diagram, even though it's more complex than a register or a wire.
The notation I've used has the `top` on the left, and separates elements with \verb':'.
So, for example, $3:4:\sigma$ has a 3 on top, a 4 in the second position, and the rest of the stack is $\sigma$.

% I had tried to have the \verb'push' signal go high immediately after the last \verb'pop' went low, but was  ...

%The operation 4+3 arises in the context of the next op being
%\verb'ADD' and with a stack of the form:
%\begin{center}
%	$4:3:\sigma$
%\end{center}
%where $4$ is at the top of the stack, $3$ is next-to-top,
%and $\sigma$ represents the rest of the stack
%
%In this context the stack changes state
%\begin{center} $4:3:\sigma$ \end{center}
%\begin{center}   $3:\sigma$ \end{center}
%\begin{center}     $\sigma$ \end{center}
%\begin{center}   $7:\sigma$ \end{center}
%
%We need to have each change on a clock edge so we can pop or push.

\subsection{Unary Operations}
The \verb'INVERT' op follows this timing example.
\begin{center}
\includegraphics[scale=1.0]{inv.timing.pdf}
\end{center}
The diagram is basically identical to the binary case, but with one cycle (POP2) omitted.

\subsection{Immediate Operations}
I'm skipping the diagram \verb'PUSH_IMMEDIATE' op.
The value is immediately available, and the stack push happens analogously to the other operations,
without the POP states.
%\begin{center}
%\includegraphics[scale=1.0]{imm.timing.pdf}
%\end{center}


\section{Output}
\subsection{Tests}
There are 3 groups of tests
\begin{description}[noitemsep]
\item[roys.mem] The example program from the homework
\item[deferred.mem] compute $5!$, but do so by pushing all values onto the stack before starting multiplication
\end{description}

Unit tests.  These are self-descriptive.
\begin{description}[noitemsep]
\item[unit.add.mem] Do only a single ADD
\item[unit.and.mem] \dots
\item[unit.div.mem] \dots
\item[unit.invert.mem] \dots
\item[unit.mod.mem] \dots
\item[unit.mul.mem] \dots
\item[unit.or.mem] \dots
\item[unit.sub.mem] \dots
\end{description}

Testing error handling.
\begin{description}[noitemsep]
\item[error.div0.mem] test division by zery
\item[error.empty1.mem] test popping 1st instruction off empty stack
\item[error.empty2.mem] test popping 2nd instruction off empty stack
\item[error.full.mem] test pushing to a full stack
\end{description}
\subsection{How to read the log files}
Normally terminating computations will end with a result in the lower-right corner.
Here's the bottom of the given example code, with the result displayed both
as decimal ($276$) and binary ($0100010100$).
\begin{verbatim}
# ----+-----------------------+-----------------+-----------+------+-----------+-------
# PC  | Operation             |  State  (next)  | pop  push | top  | op1   op2 | result
#   8 |            ADD, (  0) |  FETCH (DECODE) | 0    0    |    0 |    X,   X |    x xxxxxxxxxx
#   8 |            ADD, (  0) | DECODE (  POP2) | 1    0    |    0 |    X,   X |    x xxxxxxxxxx
#   8 |            ADD, (  0) |   POP2 (  POP1) | 1    0    |  276 |    X,   0 |    x xxxxxxxxxx
#   8 |            ADD, (  0) |   POP1 (  PUSH) | 0    0    |    0 |  276,   0 |  276 0100010100
#   8 |            ADD, (  0) |   PUSH ( FETCH) | 0    1    |    0 |  276,   0 |  276 0100010100
\end{verbatim}

A computation that ends in the ERROR state will show the transition under the \verb'State (next)' column.
In the case of divide-by-zero it's \verb'DECODE ( ERROR)'.
Other errors may transition from different states to ERROR.
\begin{verbatim}
# ----+-----------------------+-----------------+-----------+------+-----------+-------
# PC  | Operation             |  State  (next)  | pop  push | top  | op1   op2 | result
#   2 |            DIV, (  0) |  FETCH (DECODE) | 0    0    |    0 |    X,   X |    x xxxxxxxxxx
#   2 |            DIV, (  0) | DECODE ( ERROR) | 1    0    |    0 |    X,   X |    x xxxxxxxxxx
\end{verbatim}

%\subsection{Etc}
%I spent a lot of time drawing timing diagrams before I started coding.
%Midway through, I wondered if I'd just wasted all that time, but when I started coding, it was a breeze.

\section{Etc}
\subsection{Improvements}
Next time you offer this assignment, drop the logical operators, add an arithmetic unary negation operator.
The only thing logic ops add is a unary operator, and they have the cost of the student supporting both bit arrays and numbers.
They also have the problem that a 10-bit input with a leading 1 gets sign-extended, which is nonsensical for bit arrays.

It would be nice to have a way to quickly run the simulator on multiple inputs.
I was able to jerry-rig something with a makefile and symlinks.
\subsection{Challenges}
I spent far too much time debugging my initial implementation.
I suspect it's simply due to lack of experience writing non-trivial code.
Perhaps this problem was aggravated by my insistance to reduce the number of states.
Regardless, I would have loved to have had another assignment between \#2 and \#3 which was simpler -- maybe a Johnson counter \verb':-)'

\subsection{Technology}
Timing diagrams were typeset with ease using the \LaTeX\ package \verb'tikz-timing'.
The ugly FSM was generated with graphviz dot.
%\end{multicols}
\end{document}
